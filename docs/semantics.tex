\documentclass[a4paper]{scrreprt}

\usepackage{inconsolata}
\usepackage[T1]{fontenc}
\usepackage[utf8]{inputenc}
\usepackage[english]{babel}

\usepackage[colorlinks]{hyperref}
\usepackage{backref}

\usepackage{amsmath}
\usepackage{amssymb}
\usepackage{mathtools}
\usepackage{ebproof}
\usepackage{xfrac}

\usepackage{listings}

\usepackage{booktabs}
\usepackage{tabularx}

\usepackage[eulermath,beramono,pdfspacing,listings]{classicthesis}
\usepackage{arsclassica}
\usepackage{microtype}

\lstset{language=Haskell,
    keywordstyle=\color{RoyalBlue},
    basicstyle=\small\ttfamily,
    commentstyle=\color{Emerald}\ttfamily,
    stringstyle=\rmfamily,
    numberstyle=\scriptsize,
    showstringspaces=false,
    breaklines=true,
    frame=lines,
    flexiblecolumns=true,
    escapeinside={£*}{*£},
    firstnumber=last,
}

\DeclarePairedDelimiter{\abs}{\lvert}{\rvert}
\DeclarePairedDelimiter{\norm}{\lVert}{\rVert}

\newcommand{\sType}[1]{\text{\bfseries #1}}
\newcommand{\sCons}[1]{\text{\bfseries #1}}
\newcommand{\mType}[2]{\text{\bfseries #1 } #2}
\newcommand{\mCons}[2]{\text{\bfseries #1 } #2}

\begin{document}

\author{Giuseppe Lomurno}
\title{Lambda Semantics}
\date{}
\maketitle

\chapter{Typing rules}

% ===== INT =====
\begin{prooftree*}
  \hypo{n \in \mathbb{Z}}
  \infer1[\scshape TyInt]{\Gamma \vdash \mCons{IntE}{n} : \sType{LInt}}
\end{prooftree*}

% ===== BOOL =====
\begin{prooftree*}
  \hypo{b \in \mathbb{B}}
  \infer1[\scshape TyBool]{\Gamma \vdash \mCons{BoolE}{b} : \sType{LBool}}
\end{prooftree*}

% ===== UNIT =====
\begin{prooftree*}
    \infer0[\scshape TyUnit]{\Gamma \vdash \sCons{UnitE} : \sType{LUnit}}
\end{prooftree*}

% ===== PRODUCT =====
\begin{prooftree*}
  \hypo{\Gamma \vdash e_1 : \tau_1}
  \hypo{\Gamma \vdash e_2 : \tau_2}
  \infer2[\scshape TyProduct]{\Gamma \vdash \mCons{Pair}{e_1 \; e_2} : \mType{LProduct}{\tau_1 \; \tau_2}}
\end{prooftree*}

% ===== COPRODUCT =====
\begin{prooftree*}
  \hypo{\Gamma \vdash e : \tau_1}
  \infer1[\scshape TyCoProductLeft]{\Gamma \vdash \mCons{LeftE}{e} : \mType{LSum}{\tau_1 \; \tau_2}}
\end{prooftree*}
\begin{prooftree*}
  \hypo{\Gamma \vdash e : \tau_2}
  \infer1[\scshape TyCoProductRight]{\Gamma \vdash \mCons{RightE}{e} : \mType{LSum}{\tau_1 \; \tau_2}}
\end{prooftree*}

% ===== VAR =====
\begin{prooftree*}
  \hypo{x : \tau \in \Gamma}
  \infer1[\scshape TyVar]{\Gamma \vdash \mCons{Var}{x} : \tau}
\end{prooftree*}

% ===== LAMBDA =====
\begin{prooftree*}
  \hypo{\Gamma,x : \tau_1 \vdash e : \tau_2}
  \infer1[\scshape TyLambda]{\Gamma \vdash \mCons{Lambda}{[\tau_1] \; e} : \mType{LArrow}{\tau_1 \; \tau_2}}
\end{prooftree*}

% ===== APP =====
\begin{prooftree*}
  \hypo{\Gamma \vdash e_1 : \mType{LArrow}{\tau_1 \; \tau_2}}
  \hypo{\Gamma \vdash e_2 : \tau_1}
  \infer2[\scshape TyApp]{\Gamma \vdash \mCons{App}{e_1 \; e_2} : \tau_2}
\end{prooftree*}

% ===== FIX =====
\begin{prooftree*}
  \hypo{\Gamma \vdash e : \mType{LArrow}{\tau \; \tau}}
  \infer1[\scshape TyFix]{\Gamma \vdash \mCons{Fix}{e} : \tau}
\end{prooftree*}

% ===== PRIMBINOP =====
\begin{prooftree*}
  \hypo{\Gamma \vdash e_1 : \tau_1}
  \hypo{\Gamma \vdash e_2 : \tau_1}
  \infer2[\scshape TyPrimBinOp]{\Gamma \vdash \mCons{binop}{e_1 \; e_2} : \tau_2}
\end{prooftree*}

% ===== PRIMOP =====
\begin{prooftree*}
  \hypo{\Gamma \vdash e : \tau_1}
  \infer1[\scshape TyPrimOp]{\Gamma \vdash \mCons{op}{e} : \tau_2}
\end{prooftree*}

% ===== COND =====
\begin{prooftree*}
  \hypo{\Gamma \vdash e_1 : \sType{LBool}}
  \hypo{\Gamma \vdash e_2 : \tau}
  \hypo{\Gamma \vdash e_3 : \tau}
  \infer3[\scshape TyCond]{\Gamma \vdash \mCons{Cond}{e_1 \; e_2 \; e_3} : \tau}
\end{prooftree*}

% ===== CASE =====
\begin{prooftree*}
  \hypo{\Gamma \vdash e_1 : \mType{LSum}{\tau_1 \; \tau_2}}
  \hypo{\Gamma \vdash e_2 : \mType{LArrow}{\tau_1 \; \tau_3}}
  \hypo{\Gamma \vdash e_3 : \mType{LArrow}{\tau_2 \; \tau_3}}
  \infer3[\scshape TyCase]{\Gamma \vdash \mCons{Case}{e_1 \; e_2 \; e_3}: \tau_3}
\end{prooftree*}

\begin{table}[ht]
  \centering
  \begin{tabular}[t]{lll}
    binop      & $\tau_1$ & $\tau_2$ \\
    \midrule
    PrimAdd    & LInt     & LInt     \\
    PrimSub    & LInt     & LInt     \\
    PrimMul    & LInt     & LInt     \\
    PrimDiv    & LInt     & LInt     \\
    PrimIntEq  & LInt     & LBool    \\
    PrimBoolEq & LBool    & LBool    \\
    PrimAnd    & LBool    & LBool    \\
    PrimOr     & LBool    & LBool    \\
  \end{tabular}
  \quad
  \centering
  \begin{tabular}[t]{lll}
    op      & $\tau_1$  & $\tau_2$ \\
    \midrule
    PrimNeg & LInt      & LInt     \\
    PrimNot & LBool     & LBool    \\
    PrimFst & LProduct a b & a        \\
    PrimSnd & LProduct a b & b        \\
  \end{tabular}
  \caption{Arguments and result types of primitive functions.}
\end{table}

\chapter{Big-step semantics}

\begin{prooftree*}
    \infer0[\scshape SemInt]{\mCons{IntE}{v} \Downarrow v}
\end{prooftree*}

\begin{prooftree*}
  \infer0[\scshape SemBool]{\mCons{BoolE}{v} \Downarrow v}
\end{prooftree*}

\begin{prooftree*}
  \infer0[\scshape SemUnit]{\sCons{UnitE} \Downarrow ()}
\end{prooftree*}

\begin{prooftree*}
  \hypo{e_1 \Downarrow v_1}
  \hypo{e_2 \Downarrow v_2}
  \infer2[\scshape SemProduct]{\mCons{Pair}{e_1 \; e_2} \Downarrow (v_1,v_2)}
\end{prooftree*}

\begin{prooftree*}
  \hypo{e \Downarrow v}
  \infer1[\scshape SemCoProductLeft]{\mCons{LeftE}{e} \Downarrow \mCons{Left}{v}}
\end{prooftree*}

\begin{prooftree*}
  \hypo{e \Downarrow v}
  \infer1[\scshape SemCoProductRight]{\mCons{RightE}{e} \Downarrow \mCons{Right}{v}}
\end{prooftree*}

\begin{prooftree*}
  \infer0[\scshape SemLambda]{\mCons{Lambda}{\_ \; e} \Downarrow v \to [\sfrac{v}{x}]e}
\end{prooftree*}

\begin{prooftree*}
  \hypo{e_1 \Downarrow f: a \to [\sfrac{a}{x}]e}
  \hypo{e_2 \Downarrow v_2}
  \hypo{f \; v_2 \Downarrow v}
  \infer3[\scshape SemApp]{\mCons{App}{e_1 \; e_2} \Downarrow v}
\end{prooftree*}

\begin{prooftree*}
  \hypo{e \Downarrow f: a \to [\sfrac{a}{x}]e'}
  \hypo{f \; (\mCons{Fix}{e}) \Downarrow v}
  \infer2[\scshape SemFix]{\mCons{Fix}{e} \Downarrow v}
\end{prooftree*}

\begin{prooftree*}
  \hypo{e_1 \Downarrow v_1}
  \hypo{e_2 \Downarrow v_2}
  \hypo{\mCons{binop}{v_1 \; v_2} = v}
  \infer3[\scshape SemPrimBinOp]{\mCons{binop}{e_1 \; e_2} \Downarrow v}
\end{prooftree*}

\begin{prooftree*}
  \hypo{e \Downarrow v}
  \hypo{\mCons{op}{v} = v'}
  \infer2[\scshape SemPrimOp]{\mCons{op}{e} \Downarrow v'}
\end{prooftree*}

\begin{prooftree*}
  \hypo{e_1 \Downarrow \sCons{True}}
  \hypo{e_2 \Downarrow v}
  \infer2[\scshape SemCondTrue]{\mCons{Cond}{e_1 \; e_2 \; e_3} \Downarrow v}
\end{prooftree*}

\begin{prooftree*}
  \hypo{e_1 \Downarrow \sCons{False}}
  \hypo{e_3 \Downarrow v}
  \infer2[\scshape SemCondFalse]{\mCons{Cond}{e_1 \; e_2 \; e_3} \Downarrow v}
\end{prooftree*}

\begin{prooftree*}
  \hypo{e_1 \Downarrow \mCons{Left}{v}}
  \hypo{e_2 \Downarrow f : a \to [\sfrac{a}{x}]e_2'}
  \hypo{f \; v \Downarrow v'}
  \infer3[\scshape SemCaseLeft]{\mCons{Case}{e_1 \; e_2 \; e_3} \Downarrow v'}
\end{prooftree*}

\begin{prooftree*}
  \hypo{e_1 \Downarrow \mCons{Right}{v}}
  \hypo{e_3 \Downarrow f : a \to [\sfrac{a}{x}]e_3'}
  \hypo{f \; v \Downarrow v'}
  \infer3[\scshape SemCaseRight]{\mCons{Case}{e_1 \; e_2 \; e_3} \Downarrow v'}
\end{prooftree*}

\begin{table}[ht]
  \centering
  \begin{tabular}[t]{ll}
    Lambda      & Haskell    \\
    \midrule
    LInt        & Int        \\
    LBool       & Bool       \\
    LUnit       & ()         \\
    LPair a b   & (a, b)     \\
    LEither a b & Either a b \\
    LFun        & AST -> AST \\
  \end{tabular}
  \caption{Concrete Haskell representation of Lambda types.}
\end{table}

%\chapter{Small-step semantics}
\[
  \begin{prooftree}[center=false]
    \infer0[\textsc{IntE}/\textsc{BoolE}]{v \longrightarrow v}
  \end{prooftree}
  \qquad
  \begin{prooftree}[center=false]
    \infer0[\textsc{Lambda}]{e \longrightarrow \mathbold{\lambda} x : \tau \ldotp e'}
  \end{prooftree}
\]
\[
  \begin{prooftree}
    \hypo{e_1 \longrightarrow e'_1}
    \infer1[\textsc{PairA1}]{\langle e_1,e_2 \rangle \longrightarrow \langle e'_1,e_2 \rangle}
  \end{prooftree}
  \qquad
  \begin{prooftree}
    \hypo{e_2 \longrightarrow e'_2}
    \infer1[\textsc{PairA2}]{\langle e_1,e_2 \rangle \longrightarrow \langle e_1,e'_2 \rangle}
  \end{prooftree}
\]
\[
  \begin{prooftree}
    \hypo{e_1 \longrightarrow e'_1}
    \infer1[\textsc{AppF}]{e_1\,e_2 \longrightarrow e'_1\,e_2}
  \end{prooftree}
  \quad
  \begin{prooftree}
    \hypo{e_2 \longrightarrow e'_2}
    \infer1[\textsc{AppV}]{(\mathbold{\lambda}x : \tau \ldotp e_1)\,e_2 \longrightarrow (\mathbold{\lambda}x : \tau \ldotp e_1)\,e'_2}
  \end{prooftree}
\]
\[
  \begin{prooftree}
    \hypo{e[\sfrac{v}{x}] \longrightarrow v'}
    \infer1[\textsc{App}]{(\mathbold{\lambda}x : \tau \ldotp e)\,v \longrightarrow v'}
  \end{prooftree}
  \quad
  \begin{prooftree}
    \hypo{e \longrightarrow e'}
    \infer1[\textsc{FixF}]{\text{\textbf{fix }} e \longrightarrow \text{\textbf{fix }} e'}
  \end{prooftree}
  \quad
  \begin{prooftree}
    \hypo{e[\sfrac{\text{\textbf{fix }} (\mathbold{\lambda}x : \tau \ldotp e)}{x}] \rightarrow e'}
    \infer1[\textsc{FixV}]{\text{\textbf{fix }} (\mathbold{\lambda}x : \tau \ldotp e) \rightarrow e'}
  \end{prooftree}
\]
\[
  \begin{prooftree}
    \hypo{c \longrightarrow c'}
    \infer1[\textsc{CondC}]{\text{\textbf{if} } c \text{ \textbf{then} } e_1 \text{ \textbf{else} } e_2 \longrightarrow \text{\textbf{if} } c' \text{ \textbf{then} } e_1 \text{ \textbf{else} } e_2}
  \end{prooftree}
\]
\begin{prooftree*}
  \hypo{e_1 \longrightarrow e'_1}
  \infer1[\textsc{CondA1}]{\text{\textbf{if} } v \text{ \textbf{then} } e_1 \text{ \textbf{else} } e_2 \longrightarrow \text{\textbf{if} } v \text{ \textbf{then} } e'_1 \text{ \textbf{else} } e_2}
\end{prooftree*}
\begin{prooftree*}
  \hypo{e_2 \longrightarrow e'_2}
  \infer1[\textsc{CondA2}]{\text{\textbf{if} } v \text{ \textbf{then} } e_1 \text{ \textbf{else} } e_2 \longrightarrow {\text{\textbf{if} } v \text{ \textbf{then} } e_1 \text{ \textbf{else} } e'_2}}
\end{prooftree*}
\[
  \begin{prooftree}
    \hypo{v \longrightarrow \text{True}}
    \infer1[\textsc{CondTrue}]{\text{\textbf{if} } v \text{ \textbf{then} } e_1 \text{ \textbf{else} } e_2 \longrightarrow e_1}
  \end{prooftree}
  \quad
  \begin{prooftree}
    \hypo{v \longrightarrow \text{False}}
    \infer1[\textsc{CondFalse}]{\text{\textbf{if} } v \text{ \textbf{then} } e_1 \text{ \textbf{else} } e_2 \longrightarrow e_2}
  \end{prooftree}
\]
\[
  \begin{prooftree}
    \hypo{e_1 \longrightarrow e'_1}
    \infer1[\textsc{PrimBinOpA1}]{e_1 \text{ \textbf{binop} } e_2 \longrightarrow e'_1 \text{ \textbf{binop} } e_2}
  \end{prooftree}
\]
\[
  \begin{prooftree}
    \hypo{e_2 \longrightarrow e'_2}
    \infer1[\textsc{PrimBinOpA1}]{e_1 \text{ \textbf{binop} } e_2 \longrightarrow e_1 \text{ \textbf{binop} } e'_2}
  \end{prooftree}
\]
\[
  \begin{prooftree}
    \infer0[\textsc{PrimBinOp}]{v_1 \text { \textbf{binop} } v_2 \longrightarrow v}
  \end{prooftree}
  \quad
  \begin{prooftree}
    \hypo{e \longrightarrow e'}
    \infer1[\textsc{PrimOpA}]{\text{\textbf{op} } e \longrightarrow \text{\textbf{op} } e'}
  \end{prooftree}
  \quad
  \begin{prooftree}
    \infer0[\textsc{PrimOp}]{\text{\textbf{op} } v \longrightarrow v'}
  \end{prooftree}
\]


\end{document}
